\chapter{Introdução} \label{chap:intro}

\section{Contexto} \label{sec:context}

Os \textit{drones}, ou veículos aéreos não tripulados são cada vez mais uma realidade nos dias que correm. A sua área de atividade é bastante extensa e tem vindo a emergir em diversos níveis, tanto a nível civil como a nível militar, incluindo escoltas militares, troca de mercadorias em algumas cidades, gestão do trânsito, fotografia aérea, segurança urbana e por aí adiante \cite{8255738}. Devido aos obstáculos naturais que podem existir nos diferentes cenários descritos em cima ou então, por interferências intencionais, a comunicação entre os UAVs é sensível a interrupções \cite{Secinti2018} e nesse sentido, é necessário encontrar medidas para tornar a comunicação mais eficiente.

Com a chegada das tecnologias robustas das redes \textit{wireless}, os UAVs equipados com \textit{transceivers} podem ser habilitados para comunicar com os nós terrestres assim como os outros UAVs \cite{Morgenthaler2012a}. Deste modo é possível um comando qualquer da estação possa chegar a unidades cada vez mais remotas.

Com o progresso dos sistemas embebidos e a tendência para a minimalização dos sistemas micro-eletrónicos é possível adquirir pequenos ou mini-UAVs a um preço reduzido. Contudo a capacidade de um mini-UAV é limitada. A colaboração e a coordenação de múltiplos UAVs pode criar um sistema muito mais capaz do que apenas um UAV. As vantagens dos sistemas multi-UAV são as seguintes:
  \begin{itemize}
    \item{Custo: O custo de aquisição e manutenção de um mini UAV é mais pequeno quando comparado aos UAVs normais;}
    \item {Escalabilidade: A utilização de um UAV normal permite apenas uma quantidade limitada de cobertura. Contudo, os sistemas multi-UAV podem assegurar a escalabilidade da operação facilmente; }
    \item{ Capacidade de Sobrevivência: Se um UAV falhar numa missão apenas controlada esse UAV, a missão ficará comprometida. Contudo, se um UAV se desativar num sistema multi-UAV, a operação pode continuar com os outros UAVs;}
    \item{\textit{Speed-up}: É demonstrado que as missões podem ser mais rápidas com um maior número de UAVs;}
    \item{ Secção transversal de radar pequena: Em vez de apenas uma secção transversal de radar grande, os sistemas multi-UAV produzem secções transversais de radar muito pequenas, o que é crucial para aplicações militares \cite{Bekmezci2013a}.}
  \end{itemize}
  
  Ao longo desta tese irá ser abordado o tema de como estender a ligação de controlo entre a estação terrena e um UAV remoto, utilizando um sistema multi-UAV em enxame.
  
\section{Objetivos}\label{sec:goals}

Um dos principais objetivos deste trabalho de investigação é especificar, desenvolver e testar um protocolo de sobreposição com base em \textit{software open source} de modo a encaminhar instruções para comandar um UAV remoto com base num encaminhamento \textit{multi-hop}. Deverá ainda ser possível configurar a rede de forma a que esta privilegie as comunicações em linha de vista e sem fio, de modo a permitir que o comando chegue sempre ao seu destino. Deve ser possível  a inserção de novos UAVs no sistema, em qualquer ocasião, de forma a melhorar a comunicação existente e caso a mesma seja recomendável.

\section{Estrutura da Dissertação}\label{sec:struct}

Para além da introdução, esta dissertação contém mais quatro capítulos.
No Capítulo~\ref{chap:sota}, é descrito o estado da arte e são
apresentados trabalhos relacionados. O foco principal irá incidir sobretudo em protocolos \textit{open source} exitentes com a finalidade encaminhar comandos de uma base para uma unidade UAV remota.
No Capítulo~\ref{chap:chap3}, será analisado todo o \textit{hardware} e \textit{software}, assim como as diferentes soluções utilizadas no desenvolvimento da solução proposta.
No Capítulo~\ref{chap:chap4} será apresentada uma análise dos resultados conseguidos recorrrendo ao teste da solução em ambiente simulado/real. 
No Capítulo~\ref{chap:concl} será analisado todo o trabalho realizado e apresentadas as conclusões e serão também discutidas as melhorias que poderão ser desenvolvidas e melhoradas no futuro.

