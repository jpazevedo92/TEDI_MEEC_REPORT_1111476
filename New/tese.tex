%% FEUP THESIS STYLE for LaTeX2e
%% how to use feupteses (portuguese version)
%%
%% FEUP, JCL & JCF, 31 Jul 2012
%%
%% PLEASE send improvements to jlopes at fe.up.pt and to jcf at fe.up.pt
%%
%%
%% Adaptation to be used in ISEP (formally authorized by the authors) made by João Araújo 1140570@isep.ipp.pt and Vitor Sousa 1140348@isep.ipp.pt
%%
%% You should note that:
%%%
%% * If you need to use SI units you should use the \SI{}{} command as shown in the example \SI{1250}{\kilo\meter\per\hour}. To use other units please check 'Adenda-SI units' available inside the project.
%%
%% * To use acronyms correctly you need to go to 'Preamble/acronyms.tex' file and add the acronym as shown in the example \addacronym{LSA}{Laboratório de Sistemas Autónomos}. The acronyms are automatically sorted in alphabetical order.
%%
%% * To use keywords correctly you need to add English and Portuguese keywords in the keywords area below in this file. The English and Portuguese keywords are automatically displayed in Abstract and Resumo, respectively.
%%
%% * To add new Figures you just need to put them inside the 'Figures' folder and when you want to "call" it you just need to use the figure name. You do not need to put the path 'Figures/puzzle', only the figure name!
%%
%%% * For other questions about the template adaptation to be used in ISEP please send an email to 1140570@isep.ipp.pt or 1140348@isep.ipp.pt :)


%%========================================
%% Commands: pdflatex tese
%%           bibtex tese
%%           makeindex tese (only if creating an index)
%%           pdflatex tese
%% Alternative:
%%          latexmk -pdf tese.tex
%%========================================

%%----------------------------------------
%% Important Packages
%%----------------------------------------
\documentclass[11pt,a4paper,twoside,openright]{report}
\usepackage[portugues,numericrefs]{feupteses}
%% Options: 
%% - portugues: titles, etc in portuguese
%% - onpaper: links are not shown (for paper versions)
%% - backrefs: include back references from bibliography to citation place
%% - numericrefs: include (numeric) plain references style
%% - alpharefs: include alpha references style

\usepackage[utf8]{inputenc}
\usepackage{siunitx} %SI units
\usepackage{amsmath} % \begin{equation*}
\sisetup{binary-units = true, per-mode=symbol} %SI units setup
\usepackage{graphicx,graphics,float}
\usepackage{enumitem}
% Acronyms
\usepackage{upgreek}
\usepackage{datatool}
\usepackage{acronym}
\usepackage{longtable,multirow,booktabs}
\usepackage{lipsum} %dummy text
\usepackage{url}
\usepackage{caption,subcaption}
\def\UrlBreaks{\do\/\do-} 
\def\UrlBreaks{\do\_\do-} %de forma a ser possível a justificação e 'breaks' dos URLs na Bibliografia

%%----------------------------------------
%% Information about the work
%%----------------------------------------
\title{Título da Dissertação}
\author{Nome do Autor}
\school{Instituto Superior de Engenharia do Porto}
\department{Departamento de Engenharia Eletrotécnica}
\degree{Mestrado em Engenharia Eletrotécnica e de Computadores}
\area{Área de Especialização em Telecomunicações}
\requirements{Este relatório satisfaz, parcialmente, os requisitos que constam da Ficha de Unidade Curricular de Tese/Dissertação, do 2.º ano, do Mestrado em Engenharia Eletrotécnica e de Computadores.}

%%----------------------------------------
%% Dates (appear in title pages)
%%----------------------------------------
\thesisdate{DAY de MONTH de YEAR}
\thesisyear{YEAR}

%%----------------------------------------
%% Copyright text
%%----------------------------------------
\copyrightnotice{Primeiro e último nome, YEAR} % Comment if not used

%%----------------------------------------
%% Keywords (appear in abstract)
%%----------------------------------------
% English Keywords
\keywords{MIMO, OFDM, MIMO-OFDM, Wireless Communication System, Communication Channel}
% Portuguese Keywords
\keywordsPT{MIMO, OFDM, MIMO-OFDM, Sistema de comunicação sem fios, Canal de Comunicação} 

%%----------------------------------------
%% Macros
%%----------------------------------------
%% TIP: if you want to define more macros, use the mymacros file to keep them
\include{Preamble/mymacros}

%%----------------------------------------
%% Figures directory
%%----------------------------------------
%% TIP: use folder 'Figures' to keep all your figures
\graphicspath{{Figures/}}

%%----------------------------------------
%% PDF identification
%%----------------------------------------
\makeatletter
\AtBeginDocument{
\hypersetup{pdftitle=\@title}
\hypersetup{pdfauthor=\@author}
\hypersetup{pdfkeywords=\@keywords}}
\makeatother

%%========================================
%% Start of document
%%========================================
\begin{document}

%%----------------------------------------
%% Candidate and supervisors info
%%----------------------------------------
\candidate{Candidato}{Nome do Candidato, \href{mailto:11111@isep.ipp.pt}{11111@isep.ipp.pt}}

\supervisor{Orientação científica}{Nome do Orientador ISEP, \href{mailto:XXX@isep.ipp.pt}{XXX@isep.ipp.pt}}

\company{Empresa}{Nome da Empresa}

\companysupervisor{Supervisão}{Nome do Supervisor Empresa, \href{mailto:YYY@empresa.com}{YYY@empresa.com}}

%%----------------------------------------
%% Signature
%%----------------------------------------
% Uncomment signature line in the final on paper version if used
%\signature 

%%----------------------------------------
%% ISEP Logo
%%----------------------------------------
% Specify cover logo (in folder ``figures'')
\logo{ISEP_logo} 

%%----------------------------------------
%% Preparation of Dissertation (if)
%%----------------------------------------
% Uncomment for additional text below the author's name (title page)
\additionalfronttext{Preparação da Dissertação} 

%%----------------------------------------
%% Preliminary materials
%%----------------------------------------
% remove unnecessary \include{} commands
\begin{Prolog}
   \include{Preamble/quote}    % initial quotation if desired
   \chapter*{Agradecimentos}
%\addcontentsline{toc}{chapter}{Agradecimentos}

Em primeiro lugar de deixar um agradecimento especial com carinho aos meus pais e irmãos, pela força que me deram durante todo este percurso e também pelo esforço por eles feito durante este meu percurso.

Ao resto da família por estarem lá sempre que foi necessário.
Agradecer também aos amigos que estiveram presentes não só nas fases mais fáceis, mas também nas fases mais difíceis, sempre com um gesto animador, em especial ao Adriano Valadar.


A todos os profissionais do ISEP, deixo os maiores agradecimentos pelos conhecimentos transmitidos durante todo o percurso não só na licenciatura como no Mestrado de Engenharia Eletrotécnica e de Computadores – ramo de Telecomunicações. 
Quero também agradecer em especial ao meu orientador científico o engenheiro Jorge Mamede, pela orientação, disponibilidade e ajuda na concretização deste projeto.
 
As recordações dos momentos passados no ISEP, serão para sempre lembradas como uma excelente parte da minha vida.


\vspace{10mm}
\begin{flushleft}
João Pedro Mesquita Azevedo
\end{flushleft}
  % the acknowledgments
  \chapter*{Resumo}
%\addcontentsline{toc}{chapter}{Resumo}

A manutenção de veículos aéreos não tripulados, em inglês UAV, depende em grande parte do controlo remoto. O alcance da operação é limitado pelo sucesso das comunicações via rádio entre o controlador e o UAV. De modo a expandir esse alcance, podem ser utilizadas redes ou enxames de UAVs de modo a definir uma \textit{mesh} com o objetivo de se poder encaminhar comandos de controlo para unidades mais distantes.

Devido aos obstáculos naturais e artificiais que existem no nosso meio ou então devido a interferências intencionais, a comunicação entre os drones é suscetível a interrupções e nesse sentido, é necessário criar soluções que consigam optimizar a comunicação evitando ao máximo as interrupções que possam ocorrer devido aos obstáculos.

Ao longo desta tese é proposta uma especificação, desenvolvimento e teste de um protocolo de sobreposição para encaminhar instruções do controlador para comandar um UAV remoto numa base \textit{multi-hop}.\newline 

\noindent\textbf{Palavras-Chave:. }%\newline\indent
\makeatletter
\@keywordsPT
\makeatother % the PT abstract
  \chapter*{Abstract}
%\addcontentsline{toc}{chapter}{Abstract}

The maintenance of unmanned aerial vehicles, called UAV, depends on the remote control. The scope of operation is limited by the success of radio communications between the controller and the UAV. In order to expand this range, networks or swarms of UAVs may be used in order to define a mesh so that control commands can be routed to more distant units.

Due to the natural and artificial obstacles that exist in our environment or due to intentional interference, communication between the drones is susceptible to interruptions and in this sense, it is necessary to create solutions that optimize the communication avoiding to the maximum the interruptions that may occur due to obstacles.

Throughout this thesis it is proposed a specification, development and testing of an overlay protocol to route controller instructions to command a remote UAV on a multi-hop basis.\newline 

\noindent\textbf{Keywords: }%\newline\indent
\makeatletter
\@keywords
\makeatother % the abstract
  \cleardoublepage
  \pdfbookmark[0]{Conteúdo}{contents}
  \tableofcontents
  \cleardoublepage
  \pdfbookmark[0]{Lista de Figuras}{Figures}
  \listoffigures
  \cleardoublepage
  \pdfbookmark[0]{Lista de Tabelas}{tables}
  \listoftables
  \cleardoublepage
  \chapter*{Acrónimos}
\chaptermark{Acrónimos}
% Acronyms autommatically appear by alphabetic order
\DTLnewdb{acronyms}
\begin{acronym}[CDA]
\addacronym{AODV}{Ad hoc On-Demand Distance Vector}
\addacronym{CD}{Controlo Direto}
\addacronym{d2d}{Drone-to-drone}
\addacronym{FANET}{Flying Ad-Hoc Network}
\addacronym{GCS}{Ground Control Station}
\addacronym{ISEP}{Instituto Superior de Engenharia do Porto}
\addacronym{MANET}{Mobile Ad-Hoc Network}
\addacronym{MPR}{MultiPoint Relaying}
\addacronym{OLSR}{Optimized Link-State Routing}
\addacronym{SDN}{Software Defined Networking}
\addacronym{SW}{Swarm}
\addacronym{RFC}{Request for Comments}
\addacronym{UAV}{Unmanned Aerial Vehicle}
\addacronym{VTOL}{Vertical Take-Off and Landing}
\addacronym{WMN}{Wireless Mesh Network}
\end{acronym}
\DTLsort{Acronym}{acronyms}

\begin{longtable}{m{2.2cm}m{12.2cm}}
\DTLforeach*{acronyms}{\thisAcronym=Acronym,\thisDesc=Description}{
\hspace{-0.4cm}\textbf{\thisAcronym} & \thisDesc\\[5pt]1}
\end{longtable}
  % the list of abbreviations used
\end{Prolog}

%%----------------------------------------
%% Body
%%----------------------------------------
\StartBody

%% TIP: use a separate file for each chapter
%%
\chapter{Introdução} \label{chap:intro}

\section{Contexto} \label{sec:context}

Os \textit{drones}, ou veículos aéreos não tripulados são cada vez mais a ser uma realidade nos dias que correm. A sua área de atividade é bastante extensa e tem vindo a emergir em diversos níveis, tanto a nível civil como a nível militar, incluindo escoltas militares, troca de mercadorias em algumas cidades, gestão do trânsito, fotografia aérea, segurança urbana e por aí adiante \cite{8255738}. Devido aos obstáculos naturais que podem existir nos diferentes cenários descritos em cima ou então, por interferências intencionais a comunicação entre os UAVs é suscetível a interrupções \cite{Secinti2018} e nesse sentido, é necessário encontrar medidas para tornar a comunicação mais eficiente.

Com o advento das tecnologias robustas das redes \textit{wireless}, os UAVs equipados com \textit{transceivers} podem ser habilitados para comunicar com os nós terrestres assim como os outros UAVs \cite{Morgenthaler2012a}. Deste modo é possível um comando qualquer da estação possa chegar a unidades cada vez mais remotas.

Com o progresso dos sistemas embebidos e a tendência para a minimalização dos sistemas micro-eletrónicos é possível adquirir pequenos ou mini-UAVs a um preço reduzido. Contudo a capacidade de um mini-UAV é limitada. A colaboração e a coordenação de múltiplos UAVs pode criar um sistema muito mais capaz do que apenas um UAV. As vantagens dos dos sistemas multi-UAV são as seguintes:
  \begin{itemize}
    \item{Custo: O custo de aquisição e manutenção de um mini UAV é mais pequeno quando comparado aos UAVs normais;}
    \item {Escalabilidade: A utilização de um UAV normal permite apenas uma quantidade limitada de aumento de cobertura. Contudo, os sistemas multi-UAV podem assegurar a escalabilidade da operação facilmente; }
    \item{ Capacidade de Sobrevivência: Se um UAV falhar numa missão apenas controlada esse UAV, a missão ficará comprometida. Contudo, se um UAV se desativar num sistema multi-UAV, a operação pode continuar com os outros UAVs;}
    \item{\textit{Speed-up}: É demonstrado que as missões podem ser mais rápidas com um maior número de UAVs;}
    \item{ Secção transversal de radar pequena: Em vez de apenas uma secção transversal de radar grande, os sistemas multi-UAV produzem secções transversais de radar muito pequenas, o que é crucial para aplicações militares \cite{Bekmezci2013a}.}
  \end{itemize}
  
  Ao longo desta tese irá ser abordado o tema de como estender a ligação de controlo entre a estação terrena e um UAV remoto, utilizando um sistema multi-UAV em enxame.
  
\section{Objetivos}\label{sec:goals}

Um dos principais objetivos deste trabalho de investigação é especificar, desenvolver e testar um protocolo de sobreposição com base em \textit{software open source} de modo a encaminhar instruções para comandar um UAV remoto com base num encaminhamento \textit{multi-hop}. Deverá ainda ser possível configurar a rede de forma a que esta privilegie as comunicações em linha de vista e sem fio, de modo a permitir que o comando chegue sempre ao seu destino. É possível  a inserção de novos UAVs no sistema, em qualquer ocasião, de forma a melhorar a comunicação existente e caso a mesma seja recomendável.

\section{Estrutura da Dissertação}\label{sec:struct}

Para além da introdução, esta dissertação contém mais quatro capítulos.
No Capítulo~\ref{chap:sota}, é descrito o estado da arte e são
apresentados trabalhos relacionados. O foco principal irá incidir sobretudo em protoco los \textit{open source} exitentes com a finalidade encaminhar comandos de uma base para uma unidade UAV remota.
No Capítulo~\ref{chap:chap3}, será analisado todo o \textit{hardware} e \textit{software}, assim como todas as diferentes soluções utilizadas no desenvolvimento da solução proposta.
No Capítulo~\ref{chap:chap4} será apresentada uma análise dos resultados conseguidosr recorrrendo ao teste da solução em ambiente simulado/real. 
No Capítulo~\ref{chap:concl} será analisado todo o trabalho realizado e apresentadas as conclusões e serão também discutidas as melhorias que poderão ser desenvolvidas e melhoradas no futuro.


\chapter{Estado da Arte} \label{chap:sota}



\chapter{Solução Proposta}\label{chap:chap3}


\chapter{Sistema Proposto}\label{chap:chap4}
\include{Chapters/chapter5}

%%----------------------------------------
%% Appendices
%%----------------------------------------
%% Comment next 2 commands if numbered appendices are not used
\appendix
%%----------------------------------------
%% Appendix 1
%%----------------------------------------

\chapter{Lorem Ipsum} \label{ap1:Lorem}


%%----------------------------------------
%% Appendix 2
%%----------------------------------------
\chapter{Lorem Ipsum} \label{ap2:Lorem}



%%----------------------------------------
%% Final materials
%%----------------------------------------

%% Bibliography
%% Comment the next command if BibTeX file not used, 
%% Assumes that bibliography is in 'myrefs.bib'
%% plainnat-pt.bst is used by default
\PrintBib{Bibliography/myrefs}

%% Index
%% Uncomment next command if index is required, 
%% don't forget to run ``makeindex tese'' command
%\PrintIndex
\end{document}