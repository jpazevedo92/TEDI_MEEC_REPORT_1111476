\chapter*{Resumo}
%\addcontentsline{toc}{chapter}{Resumo}

A manutenção de veículos aéreos não tripulados, em inglês UAV, depende do controlo remoto. O alcance da operação está limitado pelo sucesso das comunicações via rádio entre o controlador e o UAV. De modo a expandir esse alcance, podem ser utilizados redes ou enxames de UAVs de modo a definir uma \textit{mesh} com o objetivo de se poder encaminhar comandos de controlo para unidades mais distantes.

Devido aos obstáculos naturais e artificiais que existem no nosso meio ou então devido a interferências intencionais, a comunicação entre os drones é suscetível a interrupções e nesse sentido, é necessário criar soluções que consigam optimizar a comunicação evitando ao máximo as interrupções que possam ocorrer devido aos obstáculos.

Ao longo desta tese é proposta uma especificação, desenvolvimento e teste de um protocolo de sobreposição para encaminhar instruções do controlador para comandar um UAV remoto numa base \textit{multi-hop}.\newline 

\noindent\textbf{Palavras-Chave:. }%\newline\indent
\makeatletter
\@keywordsPT
\makeatother